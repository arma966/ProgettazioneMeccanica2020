\section{Conclusioni}
La progettazione del bozzello con gancio per una portata di 55 tonnellate si può ritenere conclusa. 
L'analisi SADT, impostata nel primo capitolo, è stata fondamentale per determinare i diversi passaggi da seguire durante la fase di progettazione. 
In questo modo sono stati ridotti al minimo i feed-back di revisione necessari per mantenere la coerenza tra i diversi risultati delle varie fasi di progettazione. 
I calcoli nominali eseguiti con i modelli di scienza delle costruzioni sono stati fondamentali per determinare le dimensioni di massima del sistema. 
In questo modo sono stati validati i dati presenti in lettaratura ed è stata fornita una base per iniziare la modellazione solida.

Nel calcolo agli elementi finiti sono poi state apportate ulteriori modifiche per diminuire la concentrazione degli sforzi presente sul perno della traversa.
Le modifiche hanno riguardato l'altezza della traversa stessa e l'aumento della dimensione longitudinale per permettere l'inserimento di tutte le sfere necessarie. 
A queso punto è stato necessario aumentare la distanza tra i piastroni e di conseguenza la lunghezza dell'asse. 
In questa situazione si rivela la vera efficacia della pianificazione delle analisi.
Si è scelto di partire infatti dalla verifica del gancio (e del suo gambo) e di proseguire con traversa ed asse, in modo da percorrere la dipendenza funzionale tra gli elementi. 
La modifica della traversa ha infatti comportato la modifica dell'asse, se si fosse seguita la strada invesa (ovvero verifica dell'asse e successivamente della traversa) si sarebbe stati costretti ad una ulteriore analisi sull'asse in seguito al suo allungamento. 

Effettuando una validazione finale si è riscontrata una criticità: la massa del bozzello con gancio è risultata essere notevolmente superiore rispetto ai riferimenti visti in letteratura.
Per una portata di $55$ tonnellate ci si sarebbe aspettati una massa complessiva di $700 \; kg$ mentre la massa stimata supera i $1200 \; kg$.
Si riscontra quindi un sovradimensionamento nella quantità di materiale utilizzato. 
Questo risultato non dovrebbe sorprendere, la manifattura dei bozzelli riscontrata in letteratura viene da un lungo ciclo di progettazione verifica e miglioramento del prodotto. 
Sicuramente il gancio progettato in questa sede è funzionale, con altrettanta certezza si può affermare che non sia ottimizzato. 
La presenza degli irrigidimenti e del gruppo lamone-piastrone rende il sistema molto rigido ma aumenta eccessivamente la sua massa. 
L'uso di distanziatori al posto degli irrigidimenti porterebbe probabilmente ad una notevole riduzione della massa del sistema, il bozzello sarebbe così in linea con gli altri prodotti realizzati nel settore. 
In questa sede, si considera il risultato conseguito accettabile, dato che il prodotto non è pensato per andare sul mercato ma si è ipotizzata una realizzazione artigianale.

Bisogna infine notare che nonostante non sia stato modellato, bisognerà provvedere ad inserire un dispositivo di sicurezza di antisganciamento del carico. 


