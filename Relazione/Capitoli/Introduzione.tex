\section{Introduzione}
Questa relazione presenta la progettazione di un un bozzello con gancio avente una portata nominale di 55 tonnellate. 
L'elaborato è strutturatO come segue: viene data una panoramica su quella che è l'analisi SADT impiegata per l'individuazione e l'organizzazione funzionale delle attività da svolgere. 
Segue un dimensionamento preliminare del sistema meccanico facendo rifermento ad altri modelli presenti sul mercato e ai dati presenti in letteratura. 
Successivamente viene eseguita una verifica stutturale nominale basata sui classici modelli di scienza delle costruzioni. 
L'esito di queste analisi permette di affinare ulteriormente le dimensioni in maniera iterativa fino ad ottenere dei risultati coerenti. 
A questo punto è possibile procedere con la modellazione solida attraverso il software \textit{SolidWorks} e alle annalisi agli elementi finiti eseguite con il codice FEM integrato \textit{Simulation}. 
Attraverso l'esito di queste analisi è possibile determinare le dimensioni finali del modello e validare il progetto per quanto riguarda la verifica strutturale. 